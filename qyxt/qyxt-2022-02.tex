\documentclass[a4paper,12pt]{report}
\usepackage[a4paper,tmargin=26mm,bmargin=26mm]{geometry}
\usepackage{calc}
\usepackage[dvipsnames]{xcolor}
\usepackage{amsmath,fontspec,xunicode,contour}
\usepackage{titlesec}
\usepackage{indentfirst}

\usepackage{listings,enumerate,enumitem,paralist}
\usepackage{tocloft,booktabs,longtable,tabu}

\usepackage{fontawesome5,tcolorbox,printlen}

\usepackage{datetime2}
\usepackage[hidelinks]{hyperref}
% \hypersetup{
%     colorlinks=false,
%     pdfpagemode=FullScreen
% }

\input{/home/neruthes/DEV/ntexlibs/lib/_common.tex}
\input{/home/neruthes/DEV/ntexlibs/lib/_fontsets.tex}



\usepackage[PunctStyle=plain,RubberPunctSkip=false,CJKecglue=\hskip 4pt plus 20pt,CJKglue=\hskip 0pt]{xeCJK}
\XeTeXlinebreaklocale "zh"
\XeTeXlinebreakskip = 0pt plus 1pt

% =========================================
\usepackage{fancyhdr}
\usepackage{graphicx,eso-pic}
\graphicspath{{/home/neruthes/DEV/qyxt/qyxt/.img}}

% Optional fonts: Libertinus Serif, Tinos, Literata
\setmainfont{NewComputerModern}
\setromanfont{NewComputerModern}
\setsansfont{IBM Plex Sans}
\setmonofont{JetBrains Mono NL}
\setCJKmainfont{FandolSong}
\setCJKromanfont{FandolSong}
\setCJKsansfont{Noto Sans CJK SC}
\setCJKmonofont{Noto Sans CJK SC}




\linespread{1.22}


% Some dimensions
\newcommand{\coverbgshiftx}{0mm}
\setlength{\tabulinesep}{3pt}
\setlength{\parskip}{8pt}
\setlength{\parindent}{2em}
\setlength{\fboxsep}{0pt}


\frenchspacing


\usepackage{multicol}
\setlength{\columnsep}{2em}


\newcommand{\icode}[1]{\texttt{\footnotesize#1}}


\newcommand{\ufs}[1]{\fontsize{#1pt}{#1pt-4pt}\selectfont}
\newcommand{\csauthor}[1]{{\noindent\large\sffamily\fontspec[Ligatures=TeX]{IBM Plex Sans SemiBold}#1}\linebreak}

\newcommand{\issuesetup}[3]{
	% $1=issueid $2=sn $3=date
	\hypersetup{
		colorlinks=false,
		pdftitle={群友闲谈 - #1},
		pdfpagemode=FullScreen,
	}
	\newcommand{\cachedIssueid}[0]{#1}
	\newcommand{\cachedSn}[0]{#2}
	\newcommand{\cachedDate}[0]{#3}
}
\newcommand{\cachedcoverpagetextcolor}[0]{white}
\newcommand{\maketitlepage}[3]{
	% $1=mid $2=left $3=right
	\color{\cachedcoverpagetextcolor}
	\noindent\begin{titlepage}
		\newgeometry{hmargin=13.7mm,vmargin=18mm}
		\rmfamily
		\hspace{20pt}\vspace{70pt}\par

		% Header
		\AddToShipoutPictureFG*{
			\put(120mm,248mm){%
				\begin{minipage}[b][36mm][t]{\paperwidth-120mm-14mm}
					\color{\cachedcoverpagetextcolor}
					\flushright
					\small
					\sffamily
					\fontspec[Ligatures=TeX]{IBM Plex Sans SemiBold}
					\bfseries

					\vfill\par
					\tabcolsep=0pt
					\begin{tabu}{lr}
						{期刊编号\hspace{8pt}} & {\cachedIssueid} \\
						{发行日期\hspace{8pt}} & {\cachedDate}    \\
					\end{tabu}\par
					\vspace{28pt}
				\end{minipage}%
			}%
		}
		\AddToShipoutPictureFG*{
			\put(14mm,248mm){\rule{\paperwidth-28mm}{2mm}}%
		}
		\AddToShipoutPictureFG*{
			\put(14mm,248mm){%
				\begin{minipage}[b][36mm][t]{105mm}
					\color{\cachedcoverpagetextcolor}
					\vspace{10pt}
					\rmfamily\bfseries
					{\huge\fontsize{42pt}{42pt}\selectfont 群\hspace{13pt}友\hspace{13pt}闲\hspace{13pt}谈}\par\vspace{5pt}
					\Large THE GROUP FORUM
				\end{minipage}
			}%
		}

		% Cover art as background
		\AddToShipoutPictureBG*{\put(-\coverbgshiftx,0mm){\includegraphics[height=\paperheight]{cover/\cachedIssueid a.jpg}}}

		% Left:
		\AddToShipoutPictureFG*{
			\put(12mm,20mm){
				\begin{minipage}[b][170mm][t]{99mm}
					\flushleft
					\rmfamily\fontspec[Ligatures=TeX]{Nimbus Roman}\CJKfontspec{Noto Serif CJK SC}\bfseries\large\parskip=25pt\parindent=0pt
					#2\par\hspace{1pt}
				\end{minipage}
			}%
		}
		% Right:
		\AddToShipoutPictureFG*{
			\put(\paperwidth-99mm-16.5mm,20mm){
				\begin{minipage}[b][170mm][t]{99mm}
					\flushright
					\rmfamily\fontspec[Ligatures=TeX]{Nimbus Roman}\CJKfontspec{Noto Serif CJK SC}\bfseries\large\parskip=25pt\parindent=0pt
					#3\par\hspace{1pt}
				\end{minipage}
			}%
		}

		\fontspec[Ligatures=TeX]{Nimbus Roman}
		\CJKfontspec{Noto Serif CJK SC}
		\bfseries\large
		\parskip=25pt
		\parindent=0pt
		% Middle:
		\center
		#1
	\end{titlepage}
	\newgeometry{textwidth=38em,tmargin=20mm,bmargin=20mm}
	\pagecolor{white}\color{black}
}
\newcommand{\coverfooter}[1]{
{\footnotesize\sffamily\mdseries 主办单位:畅所欲言小声哔哔\hfill 主编:Neruthes\hfill 总第\cachedSn{}期\hfill 包含{#1}篇内容}\par
}
\newcommand{\issueispreview}[1]{
	% $1=textcolor $2=bordercolor
	\AddToShipoutPictureFG*{%
		\put(10mm,115mm){%
			\rotatebox[origin=c]{50}{
				\noindent\begin{minipage}{\paperwidth}
					\center\sffamily\fontspec{Inter}
					\color{#1}
					\ufs{100}
					\noindent
					{预览版}\par
					\ufs{70}
					\noindent
					{PREVIEW ONLY}\par
				\end{minipage}%
			}
		}%
	}
}





\newcommand{\makefinalpage}[3]{
	% $1=authors $2=editors $3=copyrightyear
	\clearpage\hspace{1pt}\vfill\par
	\pagestyle{empty}
	\small

	\tabcolsep=0pt
	\noindent
	\begin{tabu}{lX}
		{\hspace{6em}} & {}                                  \\
		{Issue:}       & {\cachedIssueid{} (SN \cachedSn{})} \\
		{Date:}        & {\cachedDate}                       \\
		{Authors:}     & {#1}                                \\
		{Editors:}     & {#2}                                \\
	\end{tabu}

	\noindent
	Copyright \copyright{} #3 Various authors. All rights reserved.\\
	版权所有{#3}各个作者,保留一切权利。
}





\newcommand{\articletitle}[4]{
	% $1=tagline $2=title $3=date $4=author
	\clearpage
	\setcounter{section}{1}
	\noindent\begin{minipage}{\linewidth}
		% Tagline
		{\small\bfseries\sffamily#1}\vspace{13pt}\par
		% Title
		\Large
		% \fontspec[Ligatures=TeX]{Latin Modern Roman}
		\fontspec[Ligatures=TeX]{Nimbus Roman}
		\CJKfontspec[Ligatures=TeX]{Noto Serif CJK SC}
		#2\par
		\vspace{13pt}\par
		% Author & Date
		\fontspec[Ligatures=TeX]{FandolSong}
		\CJKfontspec[Ligatures=TeX]{FandolSong}
		{\small 作者:#4\hfill#3}\par%
		\vspace{19pt}
		\hrule\vspace{29pt}
	\end{minipage}
}
\newcommand{\editornote}[1]{%
	\noindent\begin{minipage}{360pt}%
		\sffamily\fontsize{9pt}{9pt}\selectfont#1%
	\end{minipage}\par\vspace{25pt}
}



% Article sections
\newcommand{\fontforsection}[0]{\Large\bfseries}
\newcommand{\fontforsubsection}[0]{\large\bfseries}
\newcommand{\fontforsubsubsection}[0]{\normalsize\bfseries}
\setcounter{section}{0}
\setcounter{subsection}{1}
\setcounter{subsubsection}{1}
\renewcommand{\section}[1]{
	\vskip 15pt {\noindent{\Large\bfseries\fontforsection\arabic{section}. #1}\vskip 12pt}
	\par
	\stepcounter{section}%
}
\renewcommand{\subsection}[1]{
	\stepcounter{subsection}%
	\vskip 15pt {\noindent{\large\bfseries\fontforsubsection\arabic{section}.\arabic{subsection}. #1}\vskip 12pt}
	\par
}
\renewcommand{\subsubsection}[1]{
	\stepcounter{subsubsection}%
	\vskip 15pt {\noindent{\normalsize\bfseries\fontforsubsubsection\arabic{section}.\arabic{subsection}\arabic{subsubsection}. #1}\vskip 12pt}
	\par
}









\newcommand{\talkboard}[3]{
	% $1=title $2=subtitle $3=content
	\clearpage\vfill
	% \noindent\begin{minipage}{\textwidth}
	\begin{tcolorbox}[height=\textheight-4pt,arc=0pt,colback=white,colframe=black,boxrule=0.7pt]
		\begin{minipage}[t][80pt][t]{\linewidth}
			\normalsize
			\center
			\vskip 10pt
			\textbf{\normalsize\sffamily{}杂谈空间}\par\vskip 10pt
			\textbf{\LARGE#1}\par\vskip 10pt
			\textbf{\Large\fontspec{Nimbus Roman}#2}\par
		\end{minipage}\par
		\vskip 10pt

		\center
		\ufs{12}
		\begin{minipage}{35.5em}
			\ufs{12}
			\vspace{10pt}
			\columnsep=1.5em
			\columnseprule=0.4pt

			\begin{multicols}{2}
				\parskip=7pt
				#3\par
			\end{multicols}\par
			\vskip 10pt
		\end{minipage}
	\end{tcolorbox}
	% \end{minipage}
	\vfill\clearpage
}
\newcommand{\talkboarditem}[2]{
	% $1=author $2=content
	{\ufs{9}\faIcon{pen-nib}}\hspace{6pt}{\bfseries\sffamily#1}\par%
	\nopagebreak%
	{\ufs{12}#2}\par%
	\vskip 11pt plus 11pt\vfill\par
}






\issuesetup{2022-02}{2}{2022-06-08}

\renewcommand{\cachedcoverpagetextcolor}[0]{white}
\renewcommand{\coverbgshiftx}{0mm}



\begin{document}

\maketitlepage{
	\ufs{32}\csauthor{封面故事}造物主的自我修养\\
	\ufs{21}A Creator Prepares\par
	\vfill
	\coverfooter{9}
	% \issueispreview{Goldenrod}
}{
	\vspace{10mm}
	\ufs{24}\csauthor{Sumnery}锐评黎明杀机\\
	\ufs{21}Remarks on `Dead by Daylight'\par

	\vspace{34mm}
	\ufs{24}\csauthor{杂谈空间}流行病与经济\\
	\ufs{21}Pandemic and Economy\par
}{
	\vspace{47mm}
	\ufs{24}\csauthor{Neruthes}文艺当须百花齐放\\
	\ufs{21}Popular Arts Require Diversity\par

	\vspace{34mm}
	\ufs{18}关于建设世界一流杂志的工作方针\\
	\ufs{13}Guidelines for Creating World Top-Tier Journal\par
}

% TOC
\noindent\begin{minipage}{\linewidth}
	\vspace{20mm}\par
	\center
	\noindent{\Large\bfseries 目录}\par

	\tabcolsep=0pt
	\renewcommand{\arraystretch}{1.3}
	\noindent\begin{tabular}{lr}
		{\hspace{140mm}}                        & {} \\
		{}                                      & {} \\
		{\footnotesize\sffamily 文章}           & {} \\
		\hline
		\textbf{造物主的自我修养}               & 2  \\
		\textbf{锐评黎明杀机}                   & 5  \\
		\textbf{文艺当须百花齐放}               & 8  \\
		{}                                      & {} \\
		{\footnotesize\sffamily 杂谈空间}       & {} \\
		\hline
		\textbf{流行病与经济}                   & 10 \\
		\textbf{开心一刻}                       & 11 \\
		{}                                      & {} \\
		{\footnotesize\sffamily 公文推送}       & {} \\
		\hline
		\textbf{关于建设世界一流杂志的工作方针} & 12 \\
	\end{tabular}
\end{minipage}








\articletitle{封面故事:形而上学}{造物主的自我修养\par\Large A Creator Prepares}{2022-05-27}{Neruthes}

\renewcommand{\fontforsection}[0]{\normalsize\bfseries}

\editornote{编者按:文本最初是2022年05月27日Neruthes写给安在哉的书信。}

\noindent
亲爱的安老师:

见信好。本次来信,我想要介绍,围绕早前谈及的「世界的可实现性」之问题,我的后续思考。
若将来增添篇幅著书立说,《造物主的自我修养》当属首选标题。

\section{实现的实现}

首先我们要检验上述「实现」之概念如何从人类这一物种的意识中被实现出来。
卡尔马克思早已论断,人类的历史是劳动与阶级斗争的历史。卡尔荣格也曾谈及,人对他人的认知不可避免地从自己的主观性出发。

劳动是人类演化史的主旋律。劳动将人类的先祖塑造成人类。
劳动是这样的一种活动:某种主体出于自己的主观意识,向某种原料施加某种劳动,以谋求得到某种产出。

纵使水果与猎物可以当作自然的馈赠,将这些馈赠收入囊中的过程依然是劳动。
数千年前,尚且年轻的人类文明不可避免地套用了「劳动与产出」的经验,将世界上的物质普遍地诠释为某种劳动的产出。
于是,造物主的概念被人类创造了出来——既然我手中的美酒与长矛是劳动成果,那么这个世界本身也一定是某种「伟大的存在」的劳动成果吧。

被这样的演化史塑造的人类,难以抵御「劳动与产出」这副有色眼镜的诱惑。摘除主观性的出发点是不现实的,就像晶状体是眼球不可分割的组成部分。
但是,至少我们可以认识到,我们认识世界的方式以怎样的方式受到我们的主观性的限制。

我们如今仍然要沿用「劳动与产出」的范式吗?

\section{质料与形式}

柏拉图与亚里士多德的争论言犹在耳,《雅典学院》之画作挂在每个人的灵魂里。
在这个时代,我们十分熟悉分子和原子,波函数和凝聚态也开始成为常识。但这似乎让我们更加淡忘,什么让一个水杯是水杯。

一个玻璃水杯,其质料与沙子别无二致。英特尔与台积电,也未能走得太远。
某个物品之所以合理地成为水杯,其本质在于,作者与读者的主观性达成了共识。
作者心怀「制作水杯」之目标,以沙子为原料,施加劳动;读者心怀对「水杯」的需求,审视这个劳动成果。
那么,在它被用于盛装液体之前,在它尚且停留在货架上时,它就已经充分地成为了水杯。

是否可以固执地相信沙子与水杯都只是二氧化硅?
是否可以固执地相信水杯与芯片都只是硅原子?
是否可以固执地相信芯片与铜板都只是强子?
是否可以固执地相信质子与中子都只是夸克?

若仅着眼于质料而无视形式,似乎必然走入正确的废话。
若上述的主观的固执是可以接受的,那么对正确的废话的主观的排斥同样是可以接受的。

若我们仍然试图解释物体与物体的区别、物质与物质的区别,那么我们必然需要接受,物质并非仅是质料,物质也是形式。

自然数集有无穷个成员,却只有空集这一质料;其他元素,只是这一质料的不同形式。
无数种宏观物质可以还原成一百多种原子的不同形式,但原子也只是三种强子的不同形式。
若标准模型内每一种粒子都被证明为超弦这唯一的质料的不同形式,那么物质的形式要素就会比质料要素更加重要——
所有的物质归根结底都由相同的质料构成,则区分不同物质的唯一理由是它们的形式差异。

所以,我们甚至可以更加激进地认为,形式不仅是物质的首要要素,更是物质的唯一要素。

\section{存在与创造}

先前安老师为「实现」一词给出的定义「从不存在变为存在的过程」十分恰当,值得沿用。

一些二氧化硅原子,从沙子变成水杯的过程,凝结了无差别人类劳动于其中。
人类在制造这个水杯时,究竟实现了什么?
人类以焓为手段、以熵为代价,将一坨二氧化硅变成了另一坨二氧化硅——沙子消失了,水杯被实现了。

这个水杯的质料是人造的吗?不是。硅原子和氧原子都是既有的。

这个水杯的形式是人造的吗?是,但不全是。从夸克到二氧化硅分子的部分是既有的,从二氧化硅分子到水杯的部分是人造的。

若我们激进地无视物质的质料要素,仅承认形式要素,那么我们会得到有趣的巧合——物质的人造性问题,与增值税问题十分相似。

增值税的征税对象是商品的增值过程;至于原料本身,则被认为是无价值的。
水果是果汁的原料,果汁厂需要从上游(果农)采购水果来制造果汁卖给下游(超市)。
水果价格与果汁价格之间的差异是增值部分,政府需要向果汁厂征收增值税。
但是,果农的上游是果树,上游采购费用为零。在增值税的意义上,水果是果农从虚空中掏出来的(Ex Nihilo)。

现代社会的高科技产品,远比水杯更加精密复杂。
即使芯片是台积电造的、屏幕是京东方造的、摄像头是索尼造的,一台手机仍然称得上是富士康造的。

\section{世界之存在}

一切存在,都必然是被某种主体从不存在实现出来的吗?
这个世界,是被实现出来的吗?

我的解答已经蕴含在本文的起点。
我们只是太习惯于套用人类通过劳动改造世界的经验来诠释人类开始劳动之前的世界为何处于那种状态。
我们只是太习惯于将「劳动与产出」的范式投射给虚构的造物主。









\articletitle{游戏}{锐评黎明杀机\par\Large Remarks on `Dead by Daylight'}{2022-05-30}{Sumnery}

这个游戏其实蛮神奇的,前100小时是恐怖游戏,之后就变成竞技游戏了。一个人玩是恐怖游戏。几个人玩就变成搞笑游戏了。

我觉得黎明杀机算是我玩的网游里最神奇的一个。因为在这个游戏之前,谁都没想到恐怖游戏可以这样做、非对称竞技可以这样做。

从机制来说,它把恐怖元素和PvP网游结合得非常好。

从设计来说,它的每一个屠夫都设计得非常棒,无论是外形美术、技能和机制,还是现实意义。

我说现实意义就是因为,黎明杀机里有个屠夫叫医生,原型是杨永信。这个屠夫就是当时制作组想做一个中国风屠夫,然后开了个投票,有什么牛头马面黑白无常阎王之类的。
不知道谁搞怪地加了个杨永信进去,然后有个中国姑娘专门和制作组写信联系,告诉他们网戒中心的事情,然后他们就真的做了一个会电人的医生。

一般来说黎明杀机的新DLC,都会包含一个新屠夫和一个新人类,然后都会有各自的背景故事。
杨永信的这个DLC里的人类,就是一个电竞女孩。衣服上有个玖字因为当初联系游戏制作组的女孩名字里有这个字。

人类这边其实也就那样,再多元化也多不到哪去,主要还是屠夫。
屠夫们要么是精神病,心理变态,要么是有悲惨的过去。
有比较经典的美式恐怖屠夫,也就是那种面容畸形的电锯杀人狂。也有灵异的,比如疯人院的护士、惨死的日本少女化身怨灵。
这里有意思的地方就在于,护士和怨灵这两个屠夫分别致敬了寂静岭和贞子,结果后来黎明杀机真的买到了这两个的版权,出了联动角色。

有一些屠夫的美术层面的设计非常精妙,比如西伯利亚森林里隐居的女猎手。最后的已知记录是一战时期砍死了很多德国士兵。
比如被迫害而死的智利艺术家,比如变态杀人狂韩国偶像,比如出生时连体婴被视作异端、母亲被当作女巫烧死的法国妹子。
还有比如感染瘟疫的古巴比伦女祭司,比如反社会人格的高中生混混,各种各样的屠夫。然后大多数屠夫都有自己的主题曲,也都是契合他们自己风格的。

比如有个西部枪手的屠夫,他的BGM就是西部风,但是是恐怖曲调。我第一次听到的时候真的很惊艳,没想到能这样结合恐怖元素和西部风。

那个艺术家,她的BGM就是管风琴乐,法国连体婴BGM就是香颂一样的调调。混混高中生屠夫的BGM就是重金属摇滚。
变态连环杀人狂韩国偶像的BGM就是娱乐工业流水线的pop。

现在有27个屠夫,然后最有意思的是混混高中生。设定里它不是一个屠夫,是四个混混,因为谋杀了一个便利店售货员成了一根绳上的蚂蚱。
他们用面具隐藏身份,游戏里可以通过更换皮肤的方式选用四个人。
但是,虽然可以换衣服来换不同的人,但是他们的技能都一样,这就体现出他们是一个小团体。
你还是可以通过体态、奔跑时的喘息声来区分他们,就像悬疑办案一样。
他们每个人都有自己的面具,但是并不是专属的,谁都能戴另一个人的。就好像是一群混混为了躲避追查一样混淆身份。

我当时也是没人陪我玩,但是高中就一直想玩,所以上大学以后第一个万圣节特惠就买了。

后来因为这个游戏认识了现在的朋友,又因为这个朋友认识了最好的朋友。卧槽,这么一想,黎明杀机算是我的一切的起源了属于是。

这个屠夫也很好玩,背景故事里他是个练习生,消防事故里他故意没有救自己的队员,然后自己就单飞转型了。

每一个屠夫都有自己独特的配件,可以一定程度上加强或者改变基础玩法机制,这个韩国欧巴的配件就全都是一些录音带啊唱片啊音频文件,里面是他黑暗风格的音乐。
还有他虐杀别人的时候录下来的惨叫,他觉得那是音乐。

诶不过话说回来黎明杀机的货币机制其实还好。它是这样的:每局开始前可以带4个技能,一个道具和其附属的两个配件,还有一个祭品。

所有这些装备配置都来自于一个叫血网的随机生成的系统,在里面消耗点数可以获取各类物品,但同时选择一条线路另一条可能就会被吞噬无法选择,以此增大随机性。

每个角色都有自己独立的技能和血网,但是在血网里可以点出技能传授,这样一个角色的技能就会出现在另一个角色的血网里了。

人类玩家每局游戏都可以带1个道具和其附属的2个配件1个祭品4个技能游戏里的地图也能找到道具,但你同时只能带着一个道具。
如果成功逃生了道具会保留,如果死在游戏里道具就会掉落在遗体处,可以由其他幸存者带出来。
配件和祭品无论死活都会被消耗。

好多类似的游戏嘞,就是模仿黎明杀机这样的非对称竞技。但不知道为啥还是黎明杀机好玩。

就像最近怪奇物语新一季上映了 然后之前黎明杀机出过怪奇物语版权DLC后来停售了,据说就是因为要出新一季了所以版权方收回了授权。
这样新玩家就没法买到当初怪奇物语的DLC了,小小的遗憾。

怪奇物语DLC的音乐是最最最棒的,把黎明杀机的主题旋律和怪奇物语的音乐风格结合起来了。

\href{https://youtu.be/9nLahrUTxuQ}{https://youtu.be/9nLahrUTxuQ}

新杀手:魔王。
新逃生者:南茜·惠勒,史蒂夫·哈林顿。
新地图:霍金斯国家实验室(地下设施)。

我记得当时我刚开始玩的时候也是吓得不行,但是越害怕越想玩,现在就是老树皇了。








\articletitle{政策评论}{文艺当须百花齐放\par\Large Popular Arts Require Diversity}{2022-06-07}{Neruthes}

\editornote{编者按:本文基于2022年中国高考全国甲卷的作文题目借题发挥。}

在《红楼梦》的剧情中,众人为题词一事各抒己见,各有风韵。一时间佳句繁多,热闹非凡。

文艺工作的生命力在于人民群众的日常生活。艺术源于生活、高于生活;完全脱离生活的艺术,是没有生命力的。近年来的文艺作品中,尤其是影视作品,偶尔有一些作品远离群众的生活、远离中国社会的历史记忆,与群众的思想和生活明显脱节。这些作品在社会上受到了许多批评,群众整体上不接受、不支持。这就说明一小部分文艺工作者脱离生活、脱离群众,没有走群众路线,不能或不肯走近群众。文艺工作者必须端正态度,讲群众喜闻乐见的故事,才能在新时代市场经济的文艺市场占有一席之地,才能在当代艺术殿堂占有一席之地。

人民群众的文艺需求是多元的。有着不同时代背景、成长经历、生活方式的群众,必然会偏好不同的题材、风格、形式。这要求我们的文艺工作者深入贯彻落实科学发展观,正确、全面认识我国当代社会复杂多样的审美情趣,才能发挥自身知识与技术,充分服务于广大群众的文艺需求。必须坚决杜绝跟风、站队。既不能看什么火就一窝蜂上马类似项目,大搞「山寨」;也不能在一个作品里强行杂糅当下时兴的诸多要素,大搞「杂烩」、「端水」。虽然类型是在不断创新中产生的,但是文艺作品的每种细分类型有内在逻辑,创作过程需要尊重客观规律,不能搞主观主义。

高成本文艺创作的投资机制需要完善。高质量的影视作品、游戏作品的制作,必然伴随高成本、长周期,并非个别文艺工作者能够靠个人能力和个人积蓄「用爱发电」创作。政府需要建立长效机制,扩大政策工具箱,引导影视制作企业、游戏开发企业多元化创作,鼓励各个企业深度探索自身擅长的品类和风格。一方面,必须保障企业自主经营的创作自由;另一方面,必须劝阻企业互相跟风上马、重复投资。此外,政府也需要引导相关行业的民营资本尊重文艺市场的客观规律,杜绝跟风投资、重复投资、绊脚投资。只有切实完善相关行业的投资机制,人民群众对文艺作品的多样化需求和深度需求才能同时得到充分满足。

在十九大精神的指导下,只要党和国家深化改革、建立健全相关制度,只要文艺工作者坚持四个自信、强化四个意识,积极服务人民群众的精神生活,新时代中国特色社会主义的文化繁荣必将更上一层楼。









\talkboard{流行病与经济}{Pandemic and Economy}{
	\talkboarditem{太一}{疫情极大程度上影响了经济的发展,尤其是中小企业,即使同时许多大型企业也在亏损。但是同时,疫情也催生出了同样巨大的市场:核酸检测,酒店隔离等,无数人都想来分一杯羹。\par
		从这段时间的新闻可以看出,许多机构感觉是在赌国家的动态清零政策,一日不清零,一日不放开。但是又有文件要求下调核酸检测价格,通过人为控制样本结果,以及减少实际检测量,来获得利润。\par
		隔离也是同样巨大的市场。现在还有核酸检测费用不得由医保资金支付,要求由财政出。高昂的利润总是让人铤而走险。}\par
	\pagebreak
	\talkboarditem{Neruthes}{自新冠肺炎疫情爆发以来,全国各族人民团结一致,在党和国家的领导下打赢了一场又一场可歌可泣的防疫攻坚战。在新的时代背景下,世界经济面临严肃挑战,中国也绝无可能独善其身。\par
		为切实保障广大人民群众的安居乐业,必须坚持四个自信、强化四个意识,做到讲科学、讲政治,端正态度,从科学出发开展经济政策制定工作,坚决破除「系统性风险总爆发,经济政策需要下猛药」的错误思想。\par
		退休干部尤其需要端正态度,老一辈革命家的教导「干部从群众中来到群众中去」不是一句空话。人民群众不会允许有些干部在退休后既要靠体制内人际关系输出影响力,又要回避决策工作和宣传工作的政治责任。}\par
}








\talkboard{开心一刻}{Humor Time}{
	\talkboarditem{阿福}{还是分手了 ,谢谢你。\par
		今天是5月26号,我们最终和平分手,其实从朋友到恋人我们发生了挺多事情,能走到一起也是很不容易。我喜欢你,很喜欢你,也想和你幼稚。对啊,我总问我自己为什么还能坚持,可能没有答案,我没有备胎,也不玩暧昧,我所有的脾气爱笑爱哭都给了你,我能为你做的最后一件事,竟然是走出你的人生。我一点都不后悔,更谢谢你教会我成长。\par
		我也不知道这是谁写的,挺感动的我就复制下来了,今天疯狂星期四V我168,如果可以我想吃三桶。}\par\vfill
	\talkboarditem{Jester}{0约1去开房,0到之后发现床上坐着个陌生人怀疑是不是走错了房间,正要关门时,1刚洗完澡从卫生间出来,0看到后顿时心生不满的质问道这是谁,1连忙走到他的耳边轻声道“今天肯德基疯狂星期四,买1送1”。}\par\pagebreak
	\talkboarditem{Jester}{我来说一下关于我的瓜吧。\par
		1. 首先谢谢大家对这件事情的关心。我事先根本没有想到会这样既然已经出这件事情了,那我不如直接讲清楚吧。也不是博得大家的同情什么的,只是回应一下。\par
		2. 那人所发出来的聊天记录、照片、视频、甚至那些亲昵称呼都是真的,无一造假。可我从未将这些东西流传在任何网络平台上面,至于那个人是怎么有的,我也不清楚。\par
		3. 虽然这事出了,但我没有不尊重任何人,我不认为我有错,所以我不会认错。\par
		4. 最后,想知道到底发生了什么的,微信转我88元,肯德基疯狂星期四我把故事从头到尾给你编一遍。}\par
}






\articletitle{公文推送}{关于建设世界一流杂志的工作方针\par\Large Guidelines for Creating World Top-Tier Journal}{2022-06-01}{群友闲谈杂志社编辑部}

\editornote{编者按:本文系群友闲谈杂志社编辑部2022年度1号文件(编〔2022〕1 号)。}

\noindent
全体群友、各单位:

我部曾于2022年5月27日试发行首期《群友闲谈》杂志(刊号2022-01,总第1期)。发行后,群内外各界人士反响热烈,纷纷致信编辑部。
在来信中,群众充分表达了对编辑部工作的支持,也对当前阶段我部工作中的不足提出了诸多宝贵意见。

我部虚心接纳各界群众的批评和建议。经过认真学习仔细分析,为加强后续杂志编辑工作、顺利建成世界一流杂志(简称“世一杂”),我部拟定本方针,现公布如下:

\textbf{一、加强党在组织工作中的领导作用。}

编辑部下设创建世界一流杂志办公室(简称“创一办”),Neruthes任办公室主任,太一任办公室副主任。
由创一办负责统筹协调创建世界一流杂志工作中的各项事务。

\textbf{二、深入贯彻落实党和国家关于精神文明建设的指导意见。}

只有物质文明建设和精神文明建设都搞好,物质力量和精神力量都增强,群友的物质生活和精神生活都改善,建设世界一流杂志的事业才能顺利向前推进。
编辑部要充分运用各类平台载体,组织开展形式多样的讨论交流活动,引导群友充分参与杂志编辑工作,推动四个自信、四个意识深入人心,在全群形成广泛共识。

\textbf{三、深化改革、扩大开放。}

要切实引导群友充分参与杂志,编辑部必须深化改革、扩大开放,持续改进与群友的互动交流方式,向群友提供更加便捷的投稿渠道,充分挖掘群友日常交流话题的新闻价值。

在文章之外,开设杂谈空间、开心一刻、摄影鉴赏等版块,简化投稿流程、降低投稿门槛。
让杂志内容更多元化,既有助于丰富杂志内容,又有助于满足广大群友的不同阅读偏好。








\makefinalpage{Neruthes, Sumnery, 太一, 阿福, Jester}{Neruthes}{2022}

\noindent
Cover art by Jonatan Moerman. Licensed via Unsplash.


\end{document}
